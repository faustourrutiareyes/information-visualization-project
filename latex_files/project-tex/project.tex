\documentclass[12pt, a4paper]{article}
\usepackage[a4paper, margin = 1in]{geometry}
\usepackage{apacite}
\usepackage{graphicx}
\usepackage[font=scriptsize,labelfont=bf]{caption}
\usepackage[hyphens,spaces,obeyspaces]{url}
\usepackage[dvipsnames]{xcolor}
\usepackage{CJKutf8}
\linespread{1.25}

\title{\textbf{A Study of The Relationship Between Screen Time And Life Satisfaction of Different Age Groups In Taiwan}}
\author{{\small Fausto Urrutia (B11303091) \and \small Chintara Sahelangi (B12303097) \and \small Yanisa Saengcharoensuklert (B11303085)}}
\date{\today}

\begin{document}

\maketitle

\section*{}

\par {\Huge N}owadays, smartphones have become an essential part of daily life by transforming how we connect with the world around us. In addition to enabling easier communication, it also provides a variety of applications that significantly increase convenience. These technological devices made it much simpler to access information, a variety of entertainment options, and e-commerce services. Moreover, it also enhances health and wellness by offering a wide range of applications that track exercise routine, monitor diet, manage medications and even enable virtual consultations with medical specialists.
\par As people have increasingly relied their life on smart gadgets due to its useful features, this increases concern about the potential of mobile devices addiction such as social media addiction. Social media platforms such as Facebook, TikTok, Instagram, Twitter and Youtube are highly addictive from its interactive elements such as “like”, “follow” or “comment” (Sternlicht, n.d.). Some data suggests that these engagements increase the secretion of dopamine, a neurotransmitter that boosts positive feeling involved in reward-seeking behavior (Addiction to Electronic Devices, 2023). Thus, causing individuals to spend a progressively longer amount of time on their screens.

\subsection*{Background Information and The Research Question}
\par GilPress has reported that the average daily screen time for individuals is 7 hours worldwide and it has been increased by 50 minutes a day since 2013. Gen Z has the highest average screen time of 7 hours and 18 minutes, followed by the Millennials with 6 hours and 42 minutes. Baby Bloomers use the least internet activities with less than half of Gen X’s on average six hours of utilization (GilPress, 2023). The younger the user, the longer the amount of screen time tends to be.
\par According to the reports of happiness show that in many regions, such as in Central and Eastern Europe, younger generations tend to be happier than the elderly (Happiness and Age: Summary | the World Happiness Report, 2024). In consideration of these data, it strikes us with curiosity if this is related with the higher average screen time usage of the younger generation. Therefore, we researched for more information about the correlation between average screen time usage and people’s life satisfaction across different age groups. However, there are a limited number of previous studies, especially the one regarding Taiwan. As a result, we decided to conduct this study with the question: “Are there any relationship between screen time and well-being among different age groups in Taiwan?”
\subsection*{Paper Objective}
\par This paper aims to display the data in a visual manner by creating compelling visualizations along with discussing relevant literature, analyzing relevant previous research results, and presenting the methodology used to collect, analyze, and visualize data of our analysis.

\subsection*{Hypothesis}
\par The relationship between screen time and life satisfaction can provide insights into the effects of screen time on mental health which can further be used to inform policies and interventions aimed at reducing screen time and improving overall mental well-being among different age groups.



\section{Literature Review}

\par The time spent utilizing digital devices is significantly rising around the world, especially in younger generations. This rising in screen activities also happens in Taiwan. The survey data that was published at a press conference hosted by Parenting magazine and National Taiwan Normal University shows that more than 90 percent of Taiwanese preschoolers used technological devices more than an hour a day. Three-year-olds have the highest screen time with 2 hours and 17 minutes, followed by four-years-old with almost 2 hours and five-years-old with an hour and 36 minutes (Chen & Mazzetta, 2022).
\par Life Satisfaction refers to an individual’s overall perception of their life quality. There are several factors that contribute to life satisfaction, including sociodemographic variables such as gender, well-being, career, family, household, age, lifestyle and hobbies (Malvaso & Kang, 2022).
\par Happiness is an important criterion in evaluating an individual's standard of life. The word “happiness” has a deep meaning, it is more than simply cheerful. It is a unique emotion that is priceless, incredibly wished for, but challenging to achieve (Susniene & Jurkauskas, 2009). A large amount of up to date literature has focused on developing objective techniques for analyzing life satisfaction, using socioeconomic factors associated with happiness and quality of life (Ruiz et al., 2021).
\par According to the 2024 World’s Happiness Report, Taiwan was ranked 31st falling from 27th in last year's report. Taiwanese people below 30 were the 25th-happiest globally and ranked highest in East Asia, whereas the elderly aged 60 and above were 34th in the world, behind Singapore and China. This indicates that younger people are happier than the elderly ones in Taiwan (Taipei Times, 2024).
\par Various prior studies have presented the correlation between average screen time usage and people’s life satisfaction. The previous paper using the data from 37 European countries and North America nations have documented that continuous screen usage is linked with low life satisfaction in teenagers, particularly in females (Khan et al., 2022). While the analyzed data from 2014 European Social Survey showed that adults who limited screen time were more likely to have higher life satisfaction’s level (Z. Chen et al., 2022).

\section{Information Visualization Case Analysis}

\par Information visualization aids comprehension of raw, collected data and the conclusions we can draw from them. There are guides to a successful visualization of data that help increase understanding and accessibility of findings including but not limited to its visual appeal, effectiveness, relevancy, and consistency. Over the years, the designs for charts and graphs that put information into images have evolved with how they are used. Some graphs are better suited for comparing data, some for showcasing the distribution of data, or its composition, or its correlation. Even then, it can be further classified into the nature of its axes, whether the data is analyzed over time or among several categories, the amount of variables being taken into account, etc.
\par This study’s questioning of the relationship between screen time and mental well-being isn’t the first of its kind. Many have endeavored to probe deeper into the correlation between the two and have produced their own visualizations of the data they received.
\par One such paper by Söderman (2020), delves into the connection between teenager’s screen time and life satisfaction in Finland. In the paper, the author uses a line chart to visualize the collected data (Figure 1). The line chart illustrates the relationship between internet surfing by hours per day and the reported life satisfaction among Finnish teenagers. Reviewing the drawn line, the average reader can see that for both boys and girls, their life satisfaction decreased as their screen time increased. This negative linear correlation is especially visible between the 4-5 hours and more than 5 hours interval on the x-axis. To summarize, Söderman’s paper showed that time spent surfing the internet and life satisfaction had a negative linear correlation.
\begin{figure}[h]
    \centering
    \includegraphics[width=0.8\textwidth]{images/image2.png}
    \caption{Mean scores of life satisfaction in 2014 according to the amount of time spent on Internet surfing (Söderman, 2020)}
    \label{fig:line_chart}
\end{figure}
\par However, when evaluated through the lenses of conventional principles of illustrating data, the choice of visualizing the information through a line chart is not a sound one. Line charts are commonly used when comparing data over a period, not among items. A flaw of using a line chart to convey a relationship between two variables is that it hinges on an immediate assumption of the existence of such a relationship; it may lead to a distortion of data, forcing a correlation from the graph. Consequently, a line graph is not a suitable graph for this data. Nonetheless, this case provides an insight into comparing means as a way to suggest relationships between variables.
\par Alternatively, several studies forgo using graphs and instead show descriptive statistics and significant correlations for the reader to draw their conclusions. In doing so, calculations of mean, standard deviation and skewness often visible on graphs are rather presented in a table.
\par A paper that does so was written by Lim (2023) looking into the correlation between peer relationships, life satisfaction, and Korean adolescents' reliance on smartphones. This form of visualization (Table 1), nonetheless, comes with its own issues, which is its lack of readability for the average viewer. While numbers can be read and understood, drawing conclusions from this table is difficult and would take time to parse out the information the reader should take away. Additionally, the abundance of numbers is not visually appealing and doesn’t highlight the important findings of the study. This then goes against many of the standing principles of a successful visualization. In short, choosing to showcase correlation simply through a table of numbers is the most appropriate way to visually convey information.
\begin{table}[h]
    \centering
    \includegraphics[width=0.8\textwidth]{images/image1.png}
    \caption{Correlations and Descriptive Statistics for Study Variables (Lim, 2023)}
    \label{fig:line_chart}
\end{table}
\par Furthermore, another aspect of this paper’s topic to consider is the comparison of different age groups. Hence, choosing to visualize our findings through a line chart would add too many variables and render the chart overstimulating and inefficient, and choosing to not employ a visualizer would make it all the more confusing for the readers. So, with all these factors in mind, we will be using a box plot to visualize our data.
\par The paper written by Thirumalai et al. (2017) applies this visualization design when comparing the smoking status of people with their age (Figure 2). From this visualization, the reader can quickly see the distributions of each item on the x-axis by the y-axis and additionally make comparisons between the three categories (here, the categories in question are: a) former smokers, b) non-smokers, c) smokers) with without clustering them together.
\begin{figure}[h]
    \centering
    \includegraphics[width=0.8\textwidth]{images/example_boxplot.jpg}
    \caption{Boxplot for Smoking status based on Age (Thirumalai et al., 2017)}
    \label{fig:line_chart}
\end{figure}
\newpage
\par Box plots (also referred to as schematic plot or box-and-whiskers plot) are an effective visualization graph as it requires little in prior data assumptions, its measures resistant to outliers and they underline important landmarks of the data (Williamson et al., 1989). As we mean to compare an increase in screen time with an increase in life satisfaction and then compare that relationship by age group, a box plot is especially useful to see the different distributions and compare them due to the little space they take up. Additionally, a box plot visualizes many of the important statistical information such as its mean, quartiles neatly and quickly, already an improvement over the use of only a table.
\par Moreover, from observing the pitfalls of using a line chart to showcase a relationship between two variables, running a regression line on the means of the data would more concretely illustrate such a correlation. A regression line is a commonly used straight/curved line (depending on the dataset) that makes use of the least squares regression formula to create a line of ‘best fit’ and depict a hypothesized relationship between variables (APA, 2024).

%METHODOLOGY
\section{Methodology}

\subsection{Data}
\subsubsection{Data Collection}

%Identify the data source and describe the data collection process. Include the data collection period, the data collection method, and the data collection instrument.
\par The data is collected from the Center for Survey Research, RCHSS, Academia Sinica and was provided to us in various data formats. The data to be used is the Digital Development Survey data which has been collected from 2002 to 2023.
For the purposes of this research question we will use the data provided for 2022 and 2023. This is because the data for 2022 and 2023 is the most recent data available and will provide the most up to date information on the relationship between screen time and mental well-being among age-groups.
\par Along with the relevant data, the original questionnaire used by the research was provided to us for use as reference (Where each question is one variable in the dataset). Therefore, we will first declare which questions from the questionnaire are the most relevant to our research question.
\par From the questionnaire, the questions most relevant to the research question are as follows (translated from Chinese):
\begin{itemize}
    \item 4. How old are you? {\footnotesize(In ranges)}
    \item 7. How  many  days  a  week  do  you  use  the  Internet? {\footnotesize(In categories)}
    \item 33.  If  0  points  means  very  dissatisfied  and  10  points  means  very  satisfied,  how  would  you  rate  your  current  life?
\end{itemize}
\par Hereafter, we will only refer to this questions as variables in the dataset, named by our research group.
The questions are renamed as variables respectively as follows: \texttt{age\_range}, \texttt{internet\_usage}, and \texttt{life\_satisfaction}.
We can extract only these columns from the dataset to use in our analysis. Using a subset of the data will allow us to focus on the most relevant variables to our research question and reduce the amount of data we need to analyze.

\subsubsection{Data Cleaning}

\par What follows is the data cleaning process, where we will remove any missing or irrelevant data from the dataset.
Firstly, the questionnaire provided is answered using numbers for each question (for example, (1) is Yes and (2) is No), the values in each column are all numerical in format.
This means that, excluding \texttt{life\_satisfaction}, we will need to align each number value to its corresponding answer in the questionnaire to make the data more readable and understandable.
To illustrate, this is Question 4 and 7 from the Academia Sinica questionnaire:

\bigskip
\footnotesize

\par 4. How old are you?
\begin{itemize}
    \item (01)  12-14  years  old
    \item (02)  15-19  years  old
    \item (03)  20-29  years  old
    \item (04)  30-39  years  old
    \item (05)  40-49  years  old
    \item (06)  50-59  years  old
    \item (07)  60-64  years  old
    \item (08)  Over  65  years  old
    \item (98)  Refuse  to  answer  [Terminate interview]
\end{itemize}

\bigskip

\par \footnotesize 7. [Ask those who have used the Internet in the past three months] How many days a week do you use the Internet?
\begin{itemize}
    \item (01) Almost every day, and the Internet time is long or the frequency is high every day
    \item (02) Almost every day, but the Internet time or frequency is not high every day
    \item (03) Four to six days a week
    \item (04) One to three days a week
    \item (05) Internet access only once in more than a week
    \item (98) Don't know/refuse to answer
\end{itemize}


%Return to normal size, write after this
\normalsize \bigskip

\par Cleaning will be done by removing any rows with missing data \footnotesize(Don't Know/Refusal to answer) \normalsize in the variables \texttt{age\_range}, \texttt{internet\_usage}, and \texttt{life\_satisfaction}.
\footnote{This is because respondents are always given the option of refusal to answer, yet, answering \textit{all} relevant questions is necessary for the analysis of the data.}
\par This removal and subsequent renaming will be done using the Python programming language and the Pandas library for data manipulation and analysis.
The original dataset contains 15,142 observations. After removing all non-answered observations we are left with 13,535 useful observations for our next step, data analysis.

\subsubsection{Data Analysis}
\par After the data cleaning process, we will proceed to the data analysis process using the Matplotlib and Seaborn libraries in Python for data visualization.
This will allow us to quickly visualize the data given its size and the number of variables we are working with.
\par From here we can follow to group each data observation in order of age group, internet usage, and reported life satisfaction.
For this visualization the most useful type of plot will be a boxplot.
More specifically, the plots will be individuallly made according to each age group.
Internet usage will serve as a descending X-axis, while life satisfaction will be the Y-axis.
For example a quick plot of the data can be seen in the following boxplot grid:
\par In the case of our data, there will be a total of 8 boxplots, one for each age group.
Internet usage will descend from 1 to 5 {\footnotesize(Highest to lowest)},
and life satisfaction will be the Y-axis from 0 to 10 {\footnotesize(Lowest to highest)}.
\par A quick visualization of the data can be seen in the following figure:

\begin{figure}[!h]
    \centering
    \includegraphics[width=1\textwidth]{images/proto_boxplot.png}
    \caption{An initial look at the structure of the visualization.}
\end{figure}

\newpage

\subsection{Additional Data}
\par There is, as well, some questions that are beyond the scope of this research question but are still relevant to the dataset either for data sorting or for future research.
The questions would be added in future editions of this document, with a mention to their addition.

\subsection{Goal of Visualization Design}

\par As seen in the previous figure there is not much to be said about the visualization design as it is still in its early stages.
The visualization must be clear, concise, and easy to understand. To achieve this, design principles such as color, size, and shape will be used to highlight the most important information.
For example in the following plot, a good use of color can be seen to differentiate between the different categories
\begin{figure}[!h]
    \centering
    \includegraphics[width=0.6\textwidth]{images/color_boxplot.jpg}
    \caption{The use of color is of great visual aid and eases understanding of the data.}
\end{figure}
\newpage
\par A clean and uncluttered layout is also a goal of the final visualization.

\begin{figure}[!h]
    \centering
    \includegraphics[width=0.6\textwidth]{images/bad_plot.png}
    \caption{Designs with no clear structure or layout can be confusing and hard to read.}
\end{figure}



\clearpage

\bibliographystyle{apacite}
\bibliography{citations}
\nocite{*}

\newpage

\section*{Appendix}
\subsection*{Example of Data Cleaning}

\begin{table}[h]
    \centering
    \resizebox{0.8\textwidth}{!}{%
        \begin{tabular}{lll}
            \multicolumn{3}{c}{Pre-cleaning Dataset}                                                                                      \\ \hline
            \multicolumn{1}{|l|}{Age Range} & \multicolumn{1}{l|}{Internet Usage Category} & \multicolumn{1}{l|}{Life Satisfaction Level} \\ \hline
            03                              & 01                                           & 8                                            \\
            05                              & 01                                           & 7                                            \\
            05                              & \textcolor{red}{98}                          & 6                                            \\
            04                              & 02                                           & 7
        \end{tabular}%
    }
\end{table}
$$\vdots$$
$$\Downarrow$$


% \usepackage{graphicx}
\begin{table}[h]
    \centering
    \resizebox{\textwidth}{!}{%
        \begin{tabular}{lll}
            \multicolumn{3}{c}{Cleaned Dataset}                                                                                           \\ \hline
            \multicolumn{1}{|l|}{Age Range} & \multicolumn{1}{l|}{Internet Usage Category} & \multicolumn{1}{l|}{Life Satisfaction Level} \\ \hline
            20-29 Years old                 & Almost every day (High frequency)            & 8                                            \\
            40-49 Years old                 & Almost every day (High frequency)            & 7                                            \\
            30-29 Years old                 & Almost every day (Low frequency)             & 7
        \end{tabular}%
    }
\end{table}
$$\vdots$$
\par {\footnotesize \textit{Note:} The \textcolor{red}{red} highlighted row is removed from the dataset due to missing data.}


\end{document}